\section{Justificaci�n y objetivos del trabajo}

La justificaci�n a este trabajo nace de la necesidad de tener herramientas que demuestren ser efectivas en solucionar los problemas en la investigaci�n m�dica y que puedan ser llevadas a la pr�ctica cl�nica diaria, ya que aunque es necesaria la investigaci�n de nuevo conocimiento y caracterizaci�n de las enfermedades, es prioritario generar mayores recursos para una correcta toma de decisiones.\par

Esta idea de formaci�n de recursos es apoyada por la alta incidencia de el s�ndrome metab�lico y sus complicaciones. El gasto p�blico desbordante es otra fuente principal que justifica el esfuerzo de este trabajo.



Por lo antes descrito, el presente trabajo cuenta con los siguientes objetivos

\begin{enumerate}
\item Clasificar las complicaciones del s�ndrome metab�lico mediante el uso del �rbol de decisi�n con variables bioqu�micas y metabol�micas.
\item Uso de �rbol de decisi�n para clasificaci�n por su facilidad de interpretaci�n.
\item Mejorar el algoritmo de decisi�n paraautomatizar el diagnostico de las complicaciones del s�ndrome metab�lico.
\item Seleccionar las caracter�sticas mas importantes mediante el m�todo de paso hacia delante, Neighborhood components analysis, y an�lisis de factores, reduciendo as� las dimensiones.
\item Generar meta caracter�sticas para incrementar la eficiencia del �rbol de decisi�n para clasificar las complicaciones del SM.
\end{enumerate}
\newpage