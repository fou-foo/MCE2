\chapter{Introducci�n}
\section{Antecedentes del S�ndrome Metab�lico}
\subsection{Definici�n}

Se puede definir s�ndrome como una agrupaci�n de hallazgos cl�nicos que pueden ocurrir conjuntamente mas que lo que se pudiera deber al azar \cite{samson2014metabolic}.  A pesar de los multiples nombres dados en el pasado, el s�ndrome metab�lico es ahora usado universalmente. La definici�n fue propuesta por un grupo de trabajo de la OMS que inici� en 1998 y termino en 1999 \cite{alberti1998definition}. 

El s�ndrome metab�lico tiene m�ltples criterios de clasificaci�n dependiendo de la sociedad o instancia internacional.  Sin embargo, sobre sale los criterios de clasificaci�n de Haller que incluyen obesidad, diabetes, hiperlipoproteinemia e h�gado graso \cite{haller1977epidermiology}; y el de Singer que incluye  estos mismos padecimientos m�s hipertensi�n \cite{singer1977diagnosis},  hoy en d�a es com�n entre las diferentes definiciones la persistencia de esos trastornos integrados como la presencia de obesidad, adiposidad abdominal o indicadores de resistencia a la insulina, metabolismo de la glucosa alterado, hipertensi�n, y dislipidemia aterog�nica. M�s adelante se describir� con profundidad la epidemiolog�a del s�ndrome metab�lico.

\subsection{Epidemiolog�a}
La prevalencia reportada del S�ndrome metab�lico varia dependiendo de la definici�n usada, edad, g�nero y estado socioecon�mico.  Sin embargo, de estudios publicados en la �ltima d�cada, se estima que cerca de una cuarta parte a un tercio de los adultos pudieran cumplir con los criterios del s�ndrome. \par

En la encuesta de nutrici�n y salud de Estados Unidos (\textbf{N}ational \textbf{H}ealth and \textbf{N}utrition \textbf{E}xamination \textbf{S}urvey (NHANES)  en 1999 a 2010, en adultos mayores de 20 a�os, la prevalencia ajustada por edad fue del 25.5\% de 1999 a 2000, disminuyendo hasta 22.9\% del 2009 al 2010 \cite{beltran2013prevalence}.
En europa el estudio  \textbf{D}iabetes \textbf{E}pidemilogy: \textbf{C}ollaborative \textbf{A}nalysis of \textbf{D}iagnostic \textbf{C}riteria in \textbf{E}urope (DECODE) incluy� datos de 9 estudios poblacionales realizados en Finlandia, Holanda, Reino Unido, Suecia, Polonia e Italia, usando los valores de corte de la \textbf{F}ederaci�n \textbf{I}nternacional de \textbf{D}iabetes (IDF) el 41\% de los hombres y 38\% de las mujeres cumplen los criterios a los 46 a 71 a�os  \cite{Gao:2008aa}.\par

La prevalencia general del s�ndrome metab�lico en M�xico se desconoce; sin embargo, una publicaci�n reciente documento una prevalencia de 72.9\% del s�ndrome en mexicanos mayores de 65 a�os \cite{Ortiz-Rodriguez:2017aa}.