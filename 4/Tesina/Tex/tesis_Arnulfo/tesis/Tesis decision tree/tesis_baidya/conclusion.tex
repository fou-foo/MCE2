\chapter{Conclusi�n y trabajo futuro}

\section{Conclusiones}

Las conclusiones de este trabajo son resumidas en la siguiente lista:

\begin{itemize}
\item Desarrollamos �rbol de Decisi�n Sensible a Costos y con Mirada Adelante Generalizado (Generalized Cost Sensitive Look Ahead Decision Tree , GCSLADT) para automatizar el diagnostico de las complicaciones del s�ndrome metab�lico.
\item Demostramos el poder de meta caracter�sticas para clasificar las complicaciones del s�ndrome metab�lico.
\item Utilizamos decision tree para clasificar las complicaciones del s�ndrome metab�lico por su interpretabilidad por los m�dicos.
\item Comparamos el rendimiento de �rbol de decisi�n con diferentes scores diagn�sticos publicados en la literatura.
\item El perfil bioqu�mico es suficiente para clasificar NAFLD con �rbol de decisi�n autom�ticamente.
\end{itemize}


\section{Trabajo futuro}

Actualmente existe mucho trabajo futuro por realizar, de entre ellos se remarcan los siguientes:
El algoritmo de �rboles de decisi�n utiliza el criterio de reducci�n de entrop�a para la partici�n de los datos. Existen otros como el Gini Index y el   voy a considerar la varianza de la clase.
\begin{itemize}
\item  Generalized Cost Sensitive Look Ahead Decision Tree (GCSLADT) funciona mejor para clasificar NAFLD, presencia de diabetes y Retinopat�a, se probara en otras enfermedades.
\item Queremos  investigar el agregar diferentes tipos de aprendizaje (Any time, High speed, ensamble) a la estructura de GCSLADT. 
\end{itemize}


 